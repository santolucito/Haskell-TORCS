\section{Case Study: TORCS}

We use the Haskell bindings to TORCS to synthesize an FRP controller for a autonomous vehicle.
Since our synthesis procedure requires implementations of the pure functions to be complete, we leave those details to machine learning.

\subsection{Specifying a Driver}

\subsection{Optimization}
As a simpler case, we first provide rough implementations of the pure functions, leaving out only the critical values such as target speed and turning radius.
We use stochastic gradient descent to find optmimal values (where the cost is the lap time) to be embedded in the pure functions.


\subsection{Full Synthesis}
In this section we demonstrate how to use a nueral network to generate the low level interpretation of the pure functions to machine learning.
In this way, we still rely on the formalized program synthesis to give us a correct by construction framework for the driver.


