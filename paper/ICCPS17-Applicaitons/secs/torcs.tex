\section{Case Study: TORCS}

We use the Haskell bindings to TORCS to synthesize an FRP controller for a autonomous vehicle.
Since our synthesis procedure requires implementations of the pure functions to be complete, we leave those details to machine learning.

\subsection{Specifying a Driver}

\subsection{Optimization}
As a simpler case, we first provide rough implementations of the pure functions, leaving out only the critical values such as target speed and turning radius.
We use stochastic gradient descent to find optimal values (where the cost is the lap time) to be embedded in the pure functions.


\subsection{Full Synthesis}
In this section we demonstrate how to use a neural network to generate the low level interpretation of the pure functions to machine learning.
In this way, we still rely on the formalized program synthesis to give us a correct by construction framework for the driver.

\subsection{Neural Networks for TORCS}

Building a autonomous vehicle for TORCS has previously explored in a number of studies~\cite{torcsNN}.
This task falls into the domain of reinforcement learning, as there is no correct driver that we are trying to model - we only try to minimize the cost.
In our case, we model the cost by the speed and damage of the car during the course of the race.
We use a discounted reward to account for the immediate effect driving decisions.

The model we are building is a stochastic policy network.
At every time step where we receive sensor information, we will probablistically choose an action according to the current weight in the model.
The model will update the probabilistic weights as it trains, so that the correct action is more likely to taken.

In this work, we do not try to introduce any better way of training a neural network for TORCS.
We just build a monolithic model that is sufficient to get around the track with reasonable finesse to compare to the structured compositional network.
Specifically, we are not interested in performance to the respective networks, but only the safety.
% we really should be using a RNN to compare to the structured network b/c of the correspondence between RNN and DFA
% but that is way too much work to implement
