\documentclass[sigconf]{acmart}


\usepackage{tikz}
\usetikzlibrary{positioning,automata}


\usepackage{graphicx}
\usepackage{amsmath}
\usepackage{xspace}
\usepackage{footnote}
\usepackage{cite}
\usepackage{amsfonts}
\usepackage{float}
%\usepackage{natbib}
\usepackage{hhline}
\usepackage{multirow}
\usepackage{amssymb}
\usepackage[hyphens]{url}
\usepackage[colorlinks,linkcolor=blue,citecolor=blue,urlcolor=blue]{hyperref}
\usepackage[hyphenbreaks]{breakurl}
\usepackage{xcolor}
\usepackage{listings}
\usepackage{mathpartir}
\usepackage{caption}
\usepackage{subcaption}


\usetikzlibrary{calc}
\usetikzlibrary{automata}
\usetikzlibrary{backgrounds}
\usetikzlibrary{decorations.pathreplacing}
\usetikzlibrary{shapes,arrows}
\usetikzlibrary{positioning}
\usetikzlibrary{shadows}
\usetikzlibrary{circuits.logic.US}


\newcommand{\fullNN}{\ensuremath{N(accel,break,rpm,speed,gear)}}
\newcommand{\fullNNAtk}{\ensuremath{N(accel,break,rpm,speed_{ATK},gear)}}
\newcommand{\isAccel}{\ensuremath{A(accel,brake)}}
\newcommand{\speedGear}{\ensuremath{R(rpm,gear)}}
\newcommand{\rpmGear}{\ensuremath{S(speed,gear)}}

\newcommand{\isUnderAttack}{\ensuremath{Atk()}}

\newcommand{\inputSig}[2]{\ensuremath{\sigma_{#1}^{#2}}}
\newcommand{\outputSig}[2]{\ensuremath{\varsigma_{#1}^{#2}}}
\newcommand{\atkBegin}{\ensuremath{a_{0}}}
\newcommand{\atkLength}{\ensuremath{a}}

\newcommand{\nats}{\ensuremath{\mathbb{N}}}
\newcommand{\signal}{\ensuremath{s}}
\newcommand{\from}{\ensuremath{\colon}}
\newcommand{\dtime}{\mbox{\ensuremath{\mathcal{T}\hspace{-2.5pt}\scalebox{0.8}{\ensuremath{\boldsymbol{i}\boldsymbol{m}\boldsymbol{e}}}}}}
\renewcommand{\to}{\ensuremath{\rightarrow}}
\newcommand{\values}{\ensuremath{\mathcal{V}}}
\newcommand{\inputs}{\ensuremath{\mathcal{S}_{I}}}
\newcommand{\outputs}{\ensuremath{\mathcal{S}_{O}}}
\newcommand{\signals}{\ensuremath{\mathcal{S}}}
\newcommand{\bool}{\ensuremath{\mathcal{B}}}
\newcommand{\mealy}{\ensuremath{\mathbb{M}}\xspace}
\newcommand{\cfm}{\ensuremath{\mathcal{M}}\xspace}
\newcommand{\term}{\ensuremath{\tau}}
\newcommand{\fterm}{\ensuremath{\term_{F}}\xspace}
\newcommand{\pterm}{\ensuremath{\term_{P}}\xspace}
\newcommand{\uterm}{\ensuremath{\term_{U}}}
\newcommand{\name}[1]{{\text{\texttt{#1}}}}
\newcommand{\terms}{\ensuremath{\mathcal{T}}\xspace}
\newcommand{\comp}{\ensuremath{\varsigma}\xspace}
\newcommand{\comps}{\ensuremath{\mathcal{C}}}
\newcommand{\init}{\ensuremath{\iota}}
\newcommand{\inits}{\ensuremath{\name{init}_{\name{s}}}\xspace}
\newcommand{\sep}{\ensuremath{\quad | \quad}}
\newcommand{\upd}[2]{\ensuremath{[\,#1 \lhd \, #2\,]}}
\newcommand{\sats}{\ensuremath{\vDash}}
\newcommand{\nsats}{\ensuremath{\nvDash}}
\newcommand{\set}[1]{\ensuremath{\{ #1 \}}}
\newcommand{\eval}{\ensuremath{\eta}\xspace}



\copyrightyear{2017}
\setcopyright{acmcopyright}
\acmConference[SCAV 2017]{2017 1st International Workshop on Safe Control of Connected and Autonomous Vehicles (SCAV 2017)}{April 2017}{Pittsburgh, PA USA}
\acmISBN{978-1-4503-4976-5/17/04}
\acmPrice{\$15.00}
\acmDOI{http://dx.doi.org/10.1145/3055378.3055385}


\begin{document}
\title{Vehicle Platooning Simulations with \\ Functional Reactive Programming}
\subtitle{An opensource library}

\author{Bernd Finkbeiner}
\orcid{1234-5678-9012}
\affiliation{%
  \institution{Saarland University}
  \state{Germany} 
}

\author{Felix Klein}
\orcid{1234-5678-9012}
\affiliation{%
  \institution{Saarland University}
  \state{Germany} 
}

\author{Ruzica Piskac}
\orcid{1234-5678-9012}
\affiliation{%
  \institution{Yale University}
  %\streetaddress{}
  \state{CT, USA} 
}

\author{Mark Santolucito}
\orcid{1234-5678-9012}
\affiliation{%
  \institution{Yale University}
  %\streetaddress{}
  \state{CT, USA} 
}

\begin{abstract}
Functional languages have provided major benefits to the verification community.
Although features such as purity, a strong type system, and computational abstractions can help guide programmers away from costly errors, these can present challenges when used in a reactive system.
Functional Reactive Programming is a paradigm that allows users the benefits of functional languages and an easy interface to a reactive environment.
We present a tool for building autonomous vehicle controllers in FRP using Haskell.
\end{abstract}

\iffalse
%
% The code below should be generated by the tool at
% http://dl.acm.org/ccs.cfm
% Please copy and paste the code instead of the example below. 
%
\begin{CCSXML}
<ccs2012>
 <concept>
  <concept_id>10010520.10010553.10010562</concept_id>
  <concept_desc>Computer systems organization~Embedded systems</concept_desc>
  <concept_significance>500</concept_significance>
 </concept>
 <concept>
  <concept_id>10010520.10010575.10010755</concept_id>
  <concept_desc>Computer systems organization~Redundancy</concept_desc>
  <concept_significance>300</concept_significance>
 </concept>
 <concept>
  <concept_id>10010520.10010553.10010554</concept_id>
  <concept_desc>Computer systems organization~Robotics</concept_desc>
  <concept_significance>100</concept_significance>
 </concept>
 <concept>
  <concept_id>10003033.10003083.10003095</concept_id>
  <concept_desc>Networks~Network reliability</concept_desc>
  <concept_significance>100</concept_significance>
 </concept>
</ccs2012>  
\end{CCSXML}

\ccsdesc[500]{Computer systems organization~Embedded systems}
\ccsdesc[300]{Computer systems organization~Redundancy}
\ccsdesc{Computer systems organization~Robotics}
\ccsdesc[100]{Networks~Network reliability}

% We no longer use \terms command
%\terms{Theory}

\keywords{FRP, Autonomous Vehicles}
\fi

\maketitle

% reset title to avoid newline
\title{Vehicle Platooning Simulations with Functional Reactive Programming}

\section{Introduction}

Autonomous vehicles are considered to be one of the most challenging
types of reactive systems currently under development. They need to
interact with a highly reactive environment. Life critical decisions
have to be made instantaneously and need to be executed at the right
point in time. However, even if the intricate task of designing such
systems is conquered, it still needs to be ensured, that the result is
also safe and reliable.



\begin{itemize}

\item explain the rise of functional programming languages, and that 
types provide intuitive reasoning about safety

\item explain that functional languages now also are used for hardware
  (embedded systems people probably don't know about that) (mention
  results like \emph{clash} or \emph{Kansas Lava})

\item for reactive systems, a proper notion of time is necessary,
  motivate on some example (synchronous vs asynchronous, real vs
  discrete), finally lead to FRP

\item motivate FRP (this is the most important part, almost sure that
  nobody of the reviews will know about; go through the very basics,
  remember these are embedded systems people)

\item we provide first example of using FRP to control autonomous
  vehicle, our approach is elegant and simple, which makes it
  resistant against errors. Furthermore it is easily extendable
  without breaking the remaining system.

\end{itemize}

\section{FRP}

What is FRP

A driver observes the environment and changes the drive state over time.
This is represented as a \textit{signal function (SF)} in Yampa.

\begin{lstlisting}
type Driver = SF (Event CarState) DriveState
\end{lstlisting}

We then can create a driver, where the functions for shifting, steering, and accelerating are provided elsewhere in the code base.

\begin{lstlisting}
myDriver :: Driver
myDriver = proc e -> do
  CarState{..} <- arr getE -< e
  g <- arr shifting -< speedX
  s <- arr steering -< (angle,trackPos)
  a <- arr gas -< (speedX,s)
  returnA -< defaultDriveState {accel = a, gear = g, steer = s}
\end{lstlisting}

One major advantage of FRP is separation of control flow and data level manipulation. 
This abstraction makes it possible to easily reason about each of the components without worrying about confounding factors from the other.
For example, the user may only be concerned with verifying that their steering control is correct.
The user then only needs to verify the \texttt{steering} function in isolation from all other code.
This is a pure function, as evidenced by the type signature, which produces a new steering value based on the angle relative to the edges of the track and position on the track (where 0 is centered between the edges, and 1 is the left wall).

\begin{lstlisting}
targetSpeed = 100

shifting :: Double -> Int
shifting s = if 
  | s > 170 -> 6
  | s > 140 -> 5
  | s > 110 -> 4
  | s > 80 -> 3
  | s > 50 -> 2
  | s <= 50 -> 1
 
steering :: (Double,Double) -> Double
steering (spd,trackPos) = let
  turns = spd*14 / pi
  centering = turns - (trackPos*0.1)
  clip x = max (-1) (min x 1)
 in
  clip centering
  
gas :: (Double,Double) -> Double
gas (speed,steer) = 
  if speed < (targetSpeed-(steer*50)) then 0.5 else 0
\end{lstlisting}


\section{\ourLib}

\subsection{Basics}

A introduction to the key types and strucutre of \ourLib.

\subsection{Case Study : Driving}

As a demonstration of the \ourLib library in use, we present a case study on a simple controller for a car on an empty race track.
We implement a controller in FRP using Yampa, as shown in Listing~\ref{lst:driver}, that can be connected to TORCS using our library, \ourLib.
This code is complete and can be run as-is with an installation of TORCS.
This controller can successfully, and with some speed and finesse, navigate a vehicle in track shown in Fig.~\ref{fig:race}.

In order to connect to the library, the top level function must have type \texttt{Driver}, a type synonym for a signal function that process sensor data and makes driving decisions.

\begin{lstlisting}
type Driver = SF CarState DriveState
\end{lstlisting}

Our controller uses ArrowLoop to keep track of the current gear of the car.
Although the gear is available as sensor data, it is illustrative to keep track locally of this state.
Additioanlly, all of the data manipulation functions are pure, and lifted via \texttt{arr}.
One major advantage of FRP this separation of dependency flow and data level manipulation. 

This abstraction makes it possible to easily reason about each of the components without worrying about confounding factors from the other.
For example, if a programmer wants to verify that the steering control is correct, it is semantically guaranteed that the only function that must be checked is \texttt{steering}.
Because of Haskell's purity, this is the only place where the steering value may be changed, significantly reducing the size of the verification problem.

\begin{lstlisting}[float,floatplacement=TR,caption=A complete basic controller in Yampa, label=lst:driver]
{-# LANGUAGE Arrows,
             MultiWayIf,
             RecordWildCards #-}
module TORCS.Example where
import TORCS.Connect
import TORCS.Types

main = startDriver myDriver

myDriver :: Driver
myDriver = proc CarState{..}  -> do
  rec 
    oldG <- iPre 0 -< g
    g <- arr shifting -< (rpm,oldG)
    s <- arr steering -< (angle,trackPos)
    a <- arr gas -< (speedX,s)
  returnA -< defaultDriveState {accel = a, gear = g, steer = s}

shifting :: (Double,Int) -> Int
shifting (rpm,g) = if 
  | rpm > 6000 -> min 6 (g+1)
  | rpm < 2000 -> max 1 (g-1)
  | otherwise  -> g
 
steering :: (Double,Double) -> Double
steering (spd,trackPos) = let
  turns = spd*14 / pi
  centering = turns - (trackPos*0.1)
  clip x = max (-1) (min x 1)
 in
  clip centering

targetSpeed = 100
gas :: (Double,Double) -> Double
gas (speed,steer) = 
  if speed < (targetSpeed-(steer*50)) then 1 else 0
\end{lstlisting}


\subsection{Multi-Vehicle Communication}

Thanks to functional language's exceptional support for parallelism, controlling multiple vehicles in a multi-threaded environment is exceedingly simple. 
In our library API, the user simply needs to use \texttt{startDrivers} rather than \texttt{startDriver}, and pass a list of \texttt{Driver} signal function that should drive together.
This can be used to race various implementations against each other, or it can be used to build a vehicle platooning controller.
In this case, the user will need to be able to simulate communication between the vehicles.

Our library provides a simple interface for simulating communication between vehicles.
In order to broadcast a message to the other vehicles in the simulation, the controller simply writes a message to the \texttt{broadcast} field of the output \texttt{DriveState}.
That message is then broadcast to all other vehicles as soon as possible, and received in the \texttt{communication} field of the input \texttt{CarState}.

We allow all vehicles in the simulation to communicate irrespective of distance and with zero packet loss.
If a user wishes to simulate unreliable communications, or distance constraints, this can be simulated on a case-by-case basis.

\section{Related Work}

TORCS has been proven to provide an expressive framework for the research community~\cite{OnievaPAMP09,conf/cig/CardamoneLL09,conf/cig/MunozGS10}. 
For instance, it has been used for formal verification of platoons~\cite{kamali2016formal}. 
None of these works have used FRP as the language for the controller.


FRP specifically has been proposed as a tool for vehicle control~\cite{kazemi2016,zou2016}, where FRP was extended to prioritize functions for timing constraints. However, due to the lack of a compatible simulator, the vehicle simulation never was implemented. 

FRP has also been used for embedded systems~\cite{helbling2016juniper} and networking~\cite{voellmy2012scalable}.
The FRP networking library took advantage of Haskell's multicore support and significantly outperformed competing tools written in C++ and Java.


To the best of our knowledge this is the first FRP-based vehicle simulator.
Although there are many bindings to various vehicle simulators, these tend to use imperative languages.
For instance, TORCS allows users to directly edit the source code and add a new car in C++.
There are also TORCS bindings for python, java, and matlab, which have been used in the SCRC competition~\cite{SCRC}.

The videogame GTA~\cite{} has also been used to train image recognition software for autonomous vehicles~\cite{}.
While GTA is professionally produced game, which has more attractive graphics and a more advanced physics engine, it has a different set of issues.
First, it is proprietary software not designed for autonomous vehicle research. 
Hence, the ability to build sensor based controllers is more restricted. 
Second, unlike TORCS, which is designed primarily as a vehicle simulator, GTA's physics engine is tuned to maximize entertainment.
Using GTA as a meaningful control simulator would still be valuable work, but we leave this to the future.

\section{Future Work}

Although FRP supports both handling of continuous and discrete behaviors, we have only addressed continuous data processing.
Furthermore, FRP is be well-suited to be used in a human-in-the-loop setting, where discrete user actions must be considered in addition to sensor data. Both areas provide a huge surface for further explorations.



\subsection*{Acknowledgments} 
Supported by the European Research Council (ERC) Grant OSARES (No.\ 683300).

\bibliographystyle{ACM-Reference-Format}
\bibliography{sigproc} 

\end{document}
