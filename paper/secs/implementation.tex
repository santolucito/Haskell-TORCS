\section{Implementation}

We provide an open source library\footnote{\url{https://github.com/santolucito/Haskell-TORCS}} for interfacing Haskell FRP programs to TORCS, The Open Racing Car Simulator, which is an existing open source vehicle simulator~\cite{torcs}.
It has been used in the Simulated Car Racing Championship competition~\cite{SCRC}.
\cmark{see this paper for a description of TORCS, we should make something like this ~\cite{xu2016experimental}.}
In these competitions, a car is controlled via a socket that sends the sensor data from the vehicle and receives and processes the driving commands.

While there are existing bindings to these socket protocols for languages such as python~\cite{snakeoil,pyscrc}, we present the first such binding for the functional programming language Haskell. 
As backend, we use the FRP library Yampa~\cite{yampa}. 
However, due to the unique programming style provided by Haskell and its purity, every other FRP library is suited as well, with respect to purely syntactic adaptions.
With the assistance of FRP, we create the  vehicle controller in a principled way that allows users to manipulated sensor data in a transparent and well structured environment.

%Although the existing tool is specifically tailored to be used with our controller within the Yampa library,
%  minor modifications can allow the tool to easily work with other FRP libraries.

%It is possible to build more lower level bindings to allow non-FRP Haskell program to interface with TORCS.
%While FRP is one of the simplest interfaces for reactive systems programming in functional languages, it is also possible to use functional language in this context without FRP.

The communication channels are realized via a hash map, using the \texttt{Data.Map} interface, from vehicle identifiers to messages.
Each vehicle is given write permissions to their unique channel, where all other vehicles have read-only permissions.
The access is mutual exclusive, which is ensured by Haskell's \texttt{MVar} implementation, a threadsafe shared memory library. % to manage these channels between the threads of the different vehicles.
