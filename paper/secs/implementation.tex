\section{Implementation}

We implemented an open source library for interfacing Haskell FRP programs to TORCS.
The library is available at \url{https://github.com/santolucito/Haskell-TORCS}.

The Open Racing Car Simulator (TORCS) is an existing open source vehicle simulator~\cite{torcs}.
It has been used in the Simulated Car Racing Championship competition~\cite{SCRC}.
For these competitions, a car can be controller via a socket that sends the sensor data from the vehicle and can receive and process the driving commands.


While there are existing bindings to these socket protocols for languages such as python~\cite{snakeoil,pyscrc}, we are the first such binding for Haskell, or any FRP library.
This is likely due in part to the unique style of programming that is required for streaming socket programming in Haskell due to its purity.
With the assistance of FRP, we solved this problem in a principled way that allows users to manipulated sensor data in a transparent and structured way.


Although the existing tool is specifically tailored to be used with our controller using the Yampa library,
  minor modifications can allow the tool to easily work with other FRP libraries.

It may also be interesting to build more lower level binding to allow non-FRP Haskell program to interface with TORCS.
While FRP is one of the simplest interfaces for reactive systems programming in functional languages, it is also possible to use functional language in this context without FRP.


The communication channels are simulated as a map from vehicle id to messages.
Each vehicle is given write permission to their unique channel, that all others have read-only permissions.
This ensures that no messages will be overwritten.
We use Haskell's \texttt{MVar}, a threadsafe shared memory library to manage these channels between the threads of the different vehicles.
