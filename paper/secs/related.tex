\section{Related Work}

TORCS has been used as a simulator for formal verification of platoons~\cite{kamali2016formal}, etc... \cmark{TODO pick a few TORCS paper from google scholar}.
None of these works have used FRP as the language for the controller.


FRP specifically has been proposed as a tool for vehicle control~\cite{kazemi2016,zou2016}.
This work extends FRP to also prioritize certain functions for timing constraints.
This has been not yet been tested on vehicle simulation, in part due to the lack of a simulator compatible with the FRP language model.

FRP has been used for embedded systems~\cite{helbling2016juniper} and networking~\cite{voellmy2012scalable}.
The FRP networking library took advantage of Haskell's multicore support and significantly outperformed competing tools written in C++ and Java.


To the best of our knowledge this is the first FRP-based vehicle simulator.
Although there are many bindings to various vehicle simulators, these tend to use imperative languages.
For instance, TORCS allows users to directly edit the source code and add a new car in C++.
There are also TORCS bindings for python, java, and matlab, which have been used in the SCRC competition~\cite{SCRC}.

The videogame GTA~\cite{} has also been used to train image recognition software for autonomous vehicles~\cite{}.
While GTA is professionally produced game, which has more attractive graphics and a more advanced physics engine, it has a different set of issues.
First, is that as proprietary software that was not designed for autonomous vehcile reserach, the ability to build sensor based controllers is more restricted. 
Furthermore, unlike TORCS, which is designed primarily as a vehicle simulator, GTA's physics engine is tuned to maximize entertainment.
Using GTA as a meaningful control simulator would still be valuable work, but we leave this to future explorations.
