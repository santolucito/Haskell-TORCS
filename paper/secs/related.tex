\section{Related Work}

TORCS has been used as a simulator for formal verification of platoons~\cite{kamali2016formal}, etc...
None of these works have used FRP as the language for the controller.


FRP specifically has been proposed as a tool for vehicle control~\cite{kazemi2016,zou2016}.
This work extends FRP to also prioritize certain functions for timing constraints.
This has been not yet been tested on vehicle simulation, no doubt in part due to the lack of a simulator compatible with Haskell.

FRP has been used for embedded systems~\cite{helbling2016juniper} and networking~\cite{voellmy2012scalable}.
The FRP networking library took advantage of Haskell's multicore support and significantly outperformed competing tools written in C++ and Java.


To the best of our knowledge this is the first FRP-based vehicle simulator.
Although there are many bindings to various vehicle simulators, these tend to use imperative languages.
For instance, TORCS allows users to direct edit the package and add a new car in C++.
There are also a few TORCS bindings for python, which have been notably used in the TORCS competition.

The videogame GTA~\cite{} has also been used to train image recognition software for autonomous vehicles~\cite{}.
While GTA is professionally produced game, which has more attractive graphics and a more advanced physics engine, this suffers from two problems.
First, is that as proprietary software, the ability to build sensor based controllers is more restricted. 
Furthermore, unlike TORCS, which is designed primarily as a vehicle simulator, GTA's physics engine is tuned to maximize entertainment.
Using GTA as a meaningful simulator would then be valuable work, but we leave this to future explorations.
