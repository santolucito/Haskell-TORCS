\section{Functional Reactive Programming}

The most common solution for the construction of reactive systems in an imperative setting are call-back frameworks, embedded into a loop.
The call-backs are either used to query the state of variables, or to change them.
This imperative approach is well suited for rapid prototyping of small systems.
However, tracing behaviors over time quickly becomes unmanageably complex for larger systems.

Functional Reactive Programming instead introduces a concrete abstraction of time that allows the programmer to safely manipulate time-varying values. 
The key abstraction is given by a \textit{signal}, providing the programmer with simple type interface:

\begin{lstlisting}
  type Signal a = Time -> a
\end{lstlisting}

\noindent Intuitively, a signal is a function that relates time to a value.
For example, the type \texttt{Signal Image} represents a video, while \texttt{Signal Steer} captures a steering wheel operated over time.
To better understand how our library works, we now introduce the basic concepts and terminology from FRP.

\subsection{Arrowized FRP}

There are many types of FRP based on different abstractions from type theory.
Expressive abstractions, such as monads, allow for complex manipulation of signal flows~\cite{van2014monadic}. 
However, for most applications they are far too expressive.
We instead focus on a FRP library, Yampa, which uses the arrow abstraction, or so called Arrowized FRP~\cite{hudak2003arrows}.
Arrows generally run faster and with little need for manual optimization~\cite{yallop2016causal}, but are fundamentally less expressive than a monadic FRP~\cite{lindley2011idioms}.
This more restrictive language is in fact a benefit, as it makes it harder for the programmer to introduce errors.
As we will see in the sequel, Yampa is still powerful enough to write complex controllers to drive an autonomous vehicle, or even to communicate with other vehicles.
At the same time, the syntax is clear and accessible enough to make for an easy introduction to the FRP paradigm.

Along with signals, Yampa also introduces the abstraction of a \textit{signal function (SF)}.
This is a transformer from one signal to another.

\begin{lstlisting}
  type SF a b = Signal a -> Signal b
\end{lstlisting}

\noindent Using the previous signals, imagine a type for a steering function, which operates based on a video stream as

\begin{lstlisting}
  turn :: SF Image Steer
\end{lstlisting}

\noindent This function processes video and uses it to decide how to steer.
We omit an implementation, as the details of the data transformation are not relevant to the structure of the FRP code.

Haskell provides special syntax for Arrowized FRP, which mimics the structure of control flow charts.
The syntax provides a composition environment, in which the programmer just manages the composition of arrow functions.
Inputs are read in from the right hand side, and piped to the left hand side (output <- function -< input).
A demonstration is given in Listing~\ref{lst:arrows}.

\begin{lstlisting}[float,floatplacement=h!,caption=Basic Arrowized FRP syntax,label=lst:arrows]
myDriver :: SF Image Steer
myDriver = proc image -> do
  basicSteer    <-     turn  -< image
  adjustedSteer <- arr avoid -< (image, basicSteer)
  returnA -< adjustedSteer
\end{lstlisting}

The example introduces 
%
\begin{lstlisting}
  avoid :: (Image, Steer) -> Steer
\end{lstlisting}
%
a pure function that adjusts the basic steering plan based on the image to avoid any obstacles.
In Listing~\ref{lst:arrows}, this \texttt{avoid} function is lifted to the signal level using:
%
\begin{lstlisting}
  arr :: (a -> b) -> SF a b
\end{lstlisting}
%
The function \texttt{turn} already is on the signal level (has a SF type). Hence, we do not need to lift it.

\subsection{Stateful FRP}

To avoid obstacles on the road, we might write an \texttt{avoid2} function as shown in Listing~\ref{lst:loop}, which requires two images to calculate the adjusted steering command. 
For this, we need a mechanism to maintain state between each processing step.
Two images would be necessary to filter noise in the image, or calculates the velocity of an approaching obstacle.
To implement it, we use an abstraction called \texttt{ArrowLoop} to  save the previous state of the image for the next processing step.
The syntax is presented in Listing~\ref{lst:loop}.
Intuitively, \texttt{ArrowLoop} gives us a recursive computation, as also indicated by the \texttt{rec} keyword\footnote{We elide the technical details for the purposes of this presentation and refer the interested reader to~\cite{paterson2001icfp}.}.


\begin{lstlisting}[float,caption=Using ArrowLoop to send feedback,label=lst:loop]
myDriver :: SF Image Steer
myDriver = proc image -> do
  rec
    oldI          <- iPre null  -< image
    basicSteer    <-      turn  -< image
    adjustedSteer <- arr avoid2 -< (image, oldI, basicSteer)
  returnA -< adjustedSteer
\end{lstlisting}

The predefined function \texttt{iPre} takes an initial state, in our case an empty image, and saves images for one time step, each time it is processed.
This way, we create a feedback loop that is then used in the updated \texttt{avoid} function.
At the same time, the \texttt{rec} keyword is used to denote a section of arrow code with  mutual dependencies\footnote{Without the keyword, there is an unresolvable dependency loop.}.



 
