\section{\ourLib}

\subsection{Basics}

We have developed a library for programming controllers for autonomous vehicles using FRP, and connecting those controllers to the TORCS vehicle simulator.
This library, \ourLib, uses Yampa as the core FRP library, though the structure can easily be adapted to any Haskell FRP library.

The core functionality of \ourLib is captured in the function \texttt{startDriver :: Driver -> IO()}.
This function will automatically connect a controller, expressed with the \texttt{Driver} type, to TORCS, which results in continuous \texttt{IO()} actions.
The user must implement a controller that will process all the data available from the sensors \texttt{CarState}, and output a commands to the vehicle, as a \texttt{DriveState}.

\begin{lstlisting}
type Driver = SF CarState DriveState
\end{lstlisting}

\subsection{Case Study : Driving}

As a demonstration of the \ourLib library in use, we present a case study on a simple controller for a car on an empty race track.
We implement a controller in FRP using Yampa, as shown in Listing~\ref{lst:driver}, that can be connected to TORCS using our library, \ourLib.
This code is complete and can be run as-is with an installation of TORCS.
This controller can successfully, and with some speed and finesse, navigate a vehicle in track shown in Fig.~\ref{fig:race}.

Our controller uses ArrowLoop to keep track of the current gear of the car.
Although the gear is available as sensor data, it is illustrative to keep track locally of this state.
In general, the ArrowLoop can be used to maintain any state that may be of interest in a future processing step.
Additionally, notice all of the data manipulation functions are pure, and lifted via \texttt{arr}.
One major advantage of FRP is this separation of dependency flow and data level manipulation. 

This abstraction makes it possible to easily reason about each of the components without worrying about confounding factors from the other.
For example, if a programmer wants to verify that the steering control is correct, it is semantically guaranteed that the only function that must be checked is \texttt{steering}.
Because of Haskell's purity, this is the only place where the steering value may be changed, significantly reducing the size of the verification problem.

\begin{lstlisting}[float,floatplacement=TR,caption=A complete basic controller in Yampa, label=lst:driver]
{-# LANGUAGE Arrows,
             MultiWayIf,
             RecordWildCards #-}
module TORCS.Example where
import TORCS.Connect
import TORCS.Types

main = startDriver myDriver

myDriver :: Driver
myDriver = proc CarState{..}  -> do
  rec 
    oldG <- iPre 0 -< g
    g <- arr shifting -< (rpm,oldG)
    s <- arr steering -< (angle,trackPos)
    a <- arr gas -< (speedX,s)
  returnA -< defaultDriveState {accel = a, gear = g, steer = s}

shifting :: (Double,Int) -> Int
shifting (rpm,g) = if 
  | rpm > 6000 -> min 6 (g+1)
  | rpm < 2000 -> max 1 (g-1)
  | otherwise  -> g
 
steering :: (Double,Double) -> Double
steering (spd,trackPos) = let
  turns = spd*14 / pi
  centering = turns - (trackPos*0.1)
  clip x = max (-1) (min x 1)
 in
  clip centering

targetSpeed = 100
gas :: (Double,Double) -> Double
gas (speed,steer) = 
  if speed < (targetSpeed-(steer*50)) then 1 else 0
\end{lstlisting}


\subsection{Multi-Vehicle Communication}

Thanks to functional language's exceptional support for parallelism, controlling multiple vehicles in a multi-threaded environment is exceedingly simple. 
In our library API, the user simply needs to use \texttt{startDrivers} rather than \texttt{startDriver}, and pass a list of \texttt{Driver} signal function that should drive together.
This can be used to race various implementations against each other, or it can be used to build a vehicle platooning controller.
In this case, the user will need to be able to simulate communication between the vehicles.

Our library provides a simple interface for simulating communication between vehicles.
In order to broadcast a message to the other vehicles in the simulation, the controller simply writes a message to the \texttt{broadcast} field of the output \texttt{DriveState}.
That message is then broadcast to all other vehicles as soon as possible, and received in the \texttt{communication} field of the input \texttt{CarState}.

We allow all vehicles in the simulation to communicate irrespective of distance and with zero packet loss.
If a user wishes to simulate unreliable communications, or distance constraints, this can be simulated on a case-by-case basis.
